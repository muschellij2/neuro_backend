%%%%%%%%%%%%%%%%%%%%%%%%%%%%%%%%%%%%%%%%%%%%%%%%
% Python Cheat Sheet
% baposter Landscape Poster
% LaTeX Template
% Version 1.0 (11/06/13)
% baposter Class Created by:
% Brian Amberg (baposter@brian-amberg.de)
% This template has been downloaded from:
% http://www.LaTeXTemplates.com
% License:
% CC BY-NC-SA 3.0 (http://creativecommons.org/licenses/by-nc-sa/3.0/)
% Edited by Michelle Cristina de Sousa Baltazar
%%%%%%%%%%%%%%%%%%%%%%%%%%%%%%%%%%%%%%%%%%%%%%%%

%----------------------------------------------------------------
%	PACKAGES AND OTHER DOCUMENT CONFIGURATIONS
%----------------------------------------------------------------

\documentclass[landscape,a0paper,fontscale=0.285]{baposter} % Adjust the font scale/size here
\input{knitr_head.tex}

\title{Neuroconductor Cheat Sheet New}
\usepackage[utf8]{inputenc}

\usepackage{graphicx} % Required for including images
\graphicspath{{figures/}} % Directory in which figures are stored

\usepackage{xcolor}
\usepackage{colortbl}
\usepackage{tabu}

\usepackage{mathtools}
%\usepackage{amsmath} % For typesetting math
\usepackage{amssymb} % Adds new symbols to be used in math mode

\usepackage{booktabs} % Top and bottom rules for tables
\usepackage{enumitem} % Used to reduce itemize/enumerate spacing
\usepackage{palatino} % Use the Palatino font
\usepackage[font=small,labelfont=bf]{caption} % Required for specifying captions to tables and figures

\usepackage{multicol} % Required for multiple columns
\setlength{\columnsep}{1.5em} % Slightly increase the space between columns
\setlength{\columnseprule}{0mm} % No horizontal rule between columns

\usepackage{tikz} % Required for flow chart
\usetikzlibrary{decorations.pathmorphing}
\usetikzlibrary{shapes,arrows} % Tikz libraries required for the flow chart in the template

\newcommand{\compresslist}{ % Define a command to reduce spacing within itemize/enumerate environments, this is used right after \begin{itemize} or \begin{enumerate}
\setlength{\itemsep}{1pt}
\setlength{\parskip}{0pt}
\setlength{\parsep}{0pt}
}

\usepackage{array}

\definecolor{lightblue}{rgb}{0.145,0.6666,1} % Defines the color used for content box headers

%%% John commands standardized
\newcommand{\code}[1]{\texttt{#1}}
\newcommand{\pkg}[1]{\emph{#1}}
\newcommand{\rlang}{\texttt{R}}


\begin{document}

\begin{poster}
{
headerborder=closed, % Adds a border around the header of content boxes
colspacing=0.8em, % Column spacing
bgColorOne=white, % Background color for the gradient on the left side of the poster
bgColorTwo=white, % Background color for the gradient on the right side of the poster
borderColor=lightblue, % Border color
headerColorOne=black, % Background color for the header in the content boxes (left side)
headerColorTwo=lightblue, % Background color for the header in the content boxes (right side)
headerFontColor=white, % Text color for the header text in the content boxes
boxColorOne=white, % Background color of the content boxes
textborder=roundedleft, % Format of the border around content boxes, can be: none, bars, coils, triangles, rectangle, rounded, roundedsmall, roundedright or faded
eyecatcher=true, % Set to false for ignoring the left logo in the title and move the title left
headerheight=0.1\textheight, % Height of the header
headershape=roundedright, % Specify the rounded corner in the content box headers, can be: rectangle, small-rounded, roundedright, roundedleft or rounded
headerfont=\Large\bf\textsc, % Large, bold and sans serif font in the headers of content boxes
%textfont={\setlength{\parindent}{1.5em}}, % Uncomment for paragraph indentation
linewidth=2pt % Width of the border lines around content boxes
}
%----------------------------------------------------------------
%	TÍTULO
%----------------------------------------------------------------
{\bf\textsc{Neuroconductor Cheat Sheet}\vspace{0.5em}} % Poster title
{\textsc{\{ N e u r o c o n d u c t o r \ \ \ \ C h e a t \ \ \ \ S h e e t \} \hspace{12pt}}}
{\textsc{John Muschelli \\ (Johns Hopkins University) \hspace{12pt}}} 

%------------------------------------------------
% BÁSICO DO PYTHON
%------------------------------------------------
\headerbox{Image Objects:}{name=objects,column=0,span=2,row=0}{

%--------------------------------------
\colorbox[HTML]{CCFFFF}{\makebox[\textwidth-2\fboxsep][l]{\bf - Types:}}
\begin{itemize}\compresslist
\item \code{nifti} (\pkg{oro.nifti}) - 3D array with header information, data in memory
\item \code{antsImage} (\pkg{ANTsR}) - C\code{++} pointer, not in memory
\item \code{niftiImage} (\pkg{RNifti}) - C\code{++} pointer, not in memory
\end{itemize}


%--------------------------------------
\colorbox[HTML]{CCFFFF}{\makebox[\textwidth-2\fboxsep][l]{\bf - Manipulation:}} \linebreak \linebreak
\code{nifti} objects\\
%\begin{tabular}{l l l}
\begin{tabular}{m{4cm} m{3cm} l}
\textbf{}\\
Comparison Operators & \code{>}, \code{>=}, \code{<}, \code{<=}, \code{==}, \code{!=} & Logical image \\
Arithmetic & \code{+}, \code{-}, \code{*}, \code{/} & Numeric image \\
In operator & \code{\%in\%} & Logical vector 
\end{tabular}

\colorbox[HTML]{CCFFFF}{\makebox[\textwidth-2\fboxsep][l]{\bf - Conversion:}} \linebreak \linebreak
How to convert {\bf to} \code{nifti} objects from:\\
\begin{tabular}{m{3cm} m{5cm} m{7cm}}
\textbf{}\\
Type & function & Description \\
\code{antsImage} & \code{extrantsr::ants2oro} & Writes out image, reads in as a \code{nifti} \\
\code{antsImage} & \code{extrantsr::ants2oro(aimg, reference = img)} & Uses the \code{img} \code{nifti} object as header, faster \\
\code{niftiImage} & \code{oro.nifti::nii2oro(aimg)} & Extracts \code{aimg} array, then copies header to \code{nifti} object 
\end{tabular}

\dotfill \newline

How to convert {\bf from} \code{nifti} objects to: \\
\begin{tabular}{m{3cm} m{5cm} m{7cm}}
\textbf{}\\
Type & function & Description \\
\code{antsImage} & \code{extrantsr::oro2ants} & Writes out image, reads in as a \code{antsImage} \\
\code{antsImage} & \code{extrantsr::oro2ants(aimg, reference = img)} & Uses the \code{img} \code{antsImage} object as header, faster \\
\code{niftiImage} & \code{oro.nifti::oro2nii(aimg)} & Writes out image, reads in as a \code{niftiImage}
\end{tabular}

\vspace{0.0em} % When there are two boxes, some whitespace may need to be added if the one on the right has more content
}
%--------------------------------------
\headerbox{Image Plotting:}{name=plots,column=0,span=2,row=0, below = objects}{

\colorbox[HTML]{CCFFFF}{\makebox[\textwidth-2\fboxsep][l]{\bf - Types of Plots:}}
%\begin{tabular}{lp{2.0cm}lp{3.0cm}|}
\begin{tabular}{m{6cm} m{9cm}}
\textbf{}\\
Function & Output \\
\code{oro.nifti::orthographic} & 3-planar view of axial, sagittal, and coronal brain \\
\code{oro.nifti::image} & Prints slices of an image, make sure \code{plot.type = "single"} for only one slice \\
\code{neurobase::ortho2} & Similar to \code{oro.nifti::orthographic}, but different defaults, cross-hairs, and terms of location \\
 & When \code{ortho2(x, y)} is called, zeros in \code{y} are set to \code{NA}, assuming \code{y} is an overlay \\
\code{neurobase::double\_ortho(x, y)} & Two side-by-side orthographic views \\
\code{neurobase::multi\_overlay} & Takes in a {\bf list}; plots one slice side-by-side from each image in the list \\
\code{oro.nifti::overlay} & Single slice from \code{x} with \code{y} slice overlaid, similar to \code{image} with \code{plot.type} 
\end{tabular}


\colorbox[HTML]{CCFFFF}{\makebox[\textwidth-2\fboxsep][l]{\bf - Operations for Better Plots:}}
%\begin{tabular}{lp{2.0cm}lp{3.0cm}|}
\begin{tabular}{m{6cm} m{9cm}}
\textbf{}\\
Function & Result \\
\code{neurobase::robust_window} & Sets values above the 99.9$^{th}$ quantile to that quantile, makes images brighter \\
\code{neurobase::dropEmptyImageDimensions} & Removes dimensions with all zeros (background).  Warning: changes image dimensions \\
& Usually pass a mask, and use the \code{other.imgs} argument \\
\end{tabular}


\vspace{0.0em} % When there are two boxes, some whitespace may need to be added if the one on the right has more content
}




%------------------------------------------------
% Listas no Python
%------------------------------------------------

\headerbox{Listas no Python}{name=results,column=2,span=1,row=0}{

\colorbox[HTML]{CCFFFF}{\makebox[\textwidth-2\fboxsep][l]{\bf - Listas no Python}}
\linebreak \\
Listas são compostas por elementos de qualquer tipo (podem ser alteradas) \linebreak \\
\begin{tabular}{@{}ll@{}}
\textbf{Manipulação de Listas no Python}\\
\multicolumn{2}{l}{\cellcolor[HTML]{DDFFFF}Criação} \\
uma\_lista = [5,3,'p',9,'e'] & cria: [5,3,'p',9,'e'] \\
\multicolumn{2}{l}{\cellcolor[HTML]{DDFFFF}Acessando} \\
uma\_lista[0] & retorna: 5 \\
\multicolumn{2}{l}{\cellcolor[HTML]{DDFFFF}Fatiando} \\
uma\_lista[1:3] & retorna: [3,'p'] \\
\multicolumn{2}{l}{\cellcolor[HTML]{DDFFFF}Comprimento} \\
len(uma\_lista) & retorna: 5 \\
\multicolumn{2}{l}{\cellcolor[HTML]{DDFFFF}count( item)} \\
\multicolumn{2}{l}{Retorna quantas vezes o item foi encontrado na lista.} \\
cont(uma\_lista('p') & retorna: 1 \\
\multicolumn{2}{l}{Pode ser usado juntamente com a função while para 'andar' pelo comprimento da lista:} \\
while x < len(uma\_lista): & retorna: [3,'p']\\
\multicolumn{2}{l}{\cellcolor[HTML]{DDFFFF}Ordenar - sort()} \\
uma\_lista.sort() & retorna: [3,5,9,'e','p'] \\
\multicolumn{2}{l}{Ordenar sem alterar a lista} \\
print(sorted(uma\_lista)) & retorna: [3,5,9,'e','p'] \\
\multicolumn{2}{l}{\cellcolor[HTML]{DDFFFF}Adicionar - append(item)} \\
uma\_lista.append(37) & retorna: [5,3,'p',9,'e',37] \\
\multicolumn{2}{l}{\cellcolor[HTML]{DDFFFF}Inserir - insert(position,item)} \\
insert(uma\_lista.append(3),200) & retorna: [5,3,200,'p',9,'e'] \\
\multicolumn{2}{l}{\cellcolor[HTML]{DDFFFF}Retornar e remover - pop(position)} \\
uma\_lista.pop() & retorna: 'e' e a lista fica [5,3,'p',9] - remove o último elemento \\
uma\_lista.pop(1) & retorna: 3 e a lista fica [5,'p',9,'e'] - remove o elemento 1 \\
\multicolumn{2}{l}{\cellcolor[HTML]{DDFFFF}Remover - remove(item)} \\
uma\_lista.remove('p') & retorna: [5,3,9,'e'] \\
\multicolumn{2}{l}{\cellcolor[HTML]{DDFFFF}Inserir} \\
uma\_lista.insert(2,'z') & retorna: [5,'z',3,'p',9,'e'] - insere na posição numerada \\
\multicolumn{2}{l}{\cellcolor[HTML]{DDFFFF}Inverter - reverse()} \\
reverse(uma\_lista) & retorna: ['e',9,'p',3,5] \\
\multicolumn{2}{l}{\cellcolor[HTML]{DDFFFF}Concatenar} \\
uma\_lista+[0] & retorna: [5,3,'p',9,'e',0] \\
uma\_lista+uma\_lista & retorna: [5,3,'p',9,'e',5,3,'p',9,'e'] \\
\multicolumn{2}{l}{\cellcolor[HTML]{DDFFFF}Encontrar} \\
9 in uma\_lista & retorna: True \\
for x in uma\_lista & retorna toda a lista, um elemento por linha \\
......print(x) &  
\end{tabular}
%------------------------------------------------
}
\end{poster}
\newpage

%%%%%%%%%%%%%%%%%%%%%%%%%%%%%%%%%%%%%%%%%%%%%%%%%%%%%%%%%%
%%%%%%%%%%%%%%%%%%    SEGUNDA PÁGINA    %%%%%%%%%%%%%%%%%%
%%%%%%%%%%%%%%%%%%%%%%%%%%%%%%%%%%%%%%%%%%%%%%%%%%%%%%%%%%

\begin{poster}
{
headerborder=closed, colspacing=0.8em, bgColorOne=white, bgColorTwo=white, borderColor=lightblue, headerColorOne=black, headerColorTwo=lightblue, 
headerFontColor=white, boxColorOne=white, textborder=roundedleft, eyecatcher=true, headerheight=0.1\textheight, headershape=roundedright, headerfont=\Large\bf\textsc, linewidth=2pt 
}
%----------------------------------------------------------------
%	TITLE SECTION 
%----------------------------------------------------------------
{\bf\textsc{Python Cheat Sheet}\vspace{0.5em}} % Poster title
{\textsc{\{ P y t h o n \ \ \ \ \ C h e a t \ \ \ \ \ S h e e t\} \hspace{12pt}}}
{\textsc{Michelle Cristina de Sousa Baltazar \\ (Universidade Federal do Triângulo Mineiro) \hspace{12pt}}} 


\end{poster}
\end{document}