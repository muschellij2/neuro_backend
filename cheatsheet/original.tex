%%%%%%%%%%%%%%%%%%%%%%%%%%%%%%%%%%%%%%%%%%%%%%%%
% Python Cheat Sheet
% baposter Landscape Poster
% LaTeX Template
% Version 1.0 (11/06/13)
% baposter Class Created by:
% Brian Amberg (baposter@brian-amberg.de)
% This template has been downloaded from:
% http://www.LaTeXTemplates.com
% License:
% CC BY-NC-SA 3.0 (http://creativecommons.org/licenses/by-nc-sa/3.0/)
% Edited by Michelle Cristina de Sousa Baltazar
%%%%%%%%%%%%%%%%%%%%%%%%%%%%%%%%%%%%%%%%%%%%%%%%

%----------------------------------------------------------------
%	PACKAGES AND OTHER DOCUMENT CONFIGURATIONS
%----------------------------------------------------------------

\documentclass[landscape,a0paper,fontscale=0.285]{baposter} % Adjust the font scale/size here
\title{Python Cheat Sheet New}
\usepackage[brazilian]{babel}
\usepackage[utf8]{inputenc}

\usepackage{graphicx} % Required for including images
\graphicspath{{figures/}} % Directory in which figures are stored

\usepackage{xcolor}
\usepackage{colortbl}
\usepackage{tabu}

\usepackage{mathtools}
%\usepackage{amsmath} % For typesetting math
\usepackage{amssymb} % Adds new symbols to be used in math mode

\usepackage{booktabs} % Top and bottom rules for tables
\usepackage{enumitem} % Used to reduce itemize/enumerate spacing
\usepackage{palatino} % Use the Palatino font
\usepackage[font=small,labelfont=bf]{caption} % Required for specifying captions to tables and figures

\usepackage{multicol} % Required for multiple columns
\setlength{\columnsep}{1.5em} % Slightly increase the space between columns
\setlength{\columnseprule}{0mm} % No horizontal rule between columns

\usepackage{tikz} % Required for flow chart
\usetikzlibrary{decorations.pathmorphing}
\usetikzlibrary{shapes,arrows} % Tikz libraries required for the flow chart in the template

\newcommand{\compresslist}{ % Define a command to reduce spacing within itemize/enumerate environments, this is used right after \begin{itemize} or \begin{enumerate}
\setlength{\itemsep}{1pt}
\setlength{\parskip}{0pt}
\setlength{\parsep}{0pt}
}

\definecolor{lightblue}{rgb}{0.145,0.6666,1} % Defines the color used for content box headers

\begin{document}

\begin{poster}
{
headerborder=closed, % Adds a border around the header of content boxes
colspacing=0.8em, % Column spacing
bgColorOne=white, % Background color for the gradient on the left side of the poster
bgColorTwo=white, % Background color for the gradient on the right side of the poster
borderColor=lightblue, % Border color
headerColorOne=black, % Background color for the header in the content boxes (left side)
headerColorTwo=lightblue, % Background color for the header in the content boxes (right side)
headerFontColor=white, % Text color for the header text in the content boxes
boxColorOne=white, % Background color of the content boxes
textborder=roundedleft, % Format of the border around content boxes, can be: none, bars, coils, triangles, rectangle, rounded, roundedsmall, roundedright or faded
eyecatcher=true, % Set to false for ignoring the left logo in the title and move the title left
headerheight=0.1\textheight, % Height of the header
headershape=roundedright, % Specify the rounded corner in the content box headers, can be: rectangle, small-rounded, roundedright, roundedleft or rounded
headerfont=\Large\bf\textsc, % Large, bold and sans serif font in the headers of content boxes
%textfont={\setlength{\parindent}{1.5em}}, % Uncomment for paragraph indentation
linewidth=2pt % Width of the border lines around content boxes
}
%----------------------------------------------------------------
%	TÍTULO
%----------------------------------------------------------------
{\bf\textsc{Python Cheat Sheet}\vspace{0.5em}} % Poster title
{\textsc{\{ P y t h o n \ \ \ \ \ C h e a t \ \ \ \ \ S h e e t\} \hspace{12pt}}}
{\textsc{Michelle Cristina de Sousa Baltazar \\ (Universidade Federal do Triângulo Mineiro) \hspace{12pt}}} 


%------------------------------------------------
% BÁSICO DO PYTHON
%------------------------------------------------
\headerbox{Básico do Python:}{name=objectives,column=0,row=0}{

%--------------------------------------
\colorbox[HTML]{CCFFFF}{\makebox[\textwidth-2\fboxsep][l]{\bf - Dicas:}}
\begin{itemize}\compresslist
\item Cuidado com espaços em branco! Eles fazem grande diferença na codificação.
\item Seu código não rodará corretamente sem a devida identação!
\item \# isto é um comentário - utilize para comentar linha a linha do código
\item ''''''\newline tudo o que estiver entre 3 aspas será considerado comentário - pode ser utilizado para textor maiores com quebra de linha \newline''''''
\end{itemize}


%--------------------------------------
\colorbox[HTML]{CCFFFF}{\makebox[\textwidth-2\fboxsep][l]{\bf - Números:}} \linebreak \linebreak
Python utiliza números inteiros e flutuantes. Pode ser utilizada a função type pra checar o valor de um objeto:\\
\begin{tabular}{l l}
\textbf{}\\
type(3) & retorna: <type 'int'> \\
type(3.14) & retorna: <type 'float'> \\
\end{tabular}

\dotfill \newline

%--------------------------------------
\colorbox[HTML]{CCFFFF}{\makebox[\textwidth-2\fboxsep][l]{\bf - Entrada de Dados:}}
%\begin{tabular}{lp{2.0cm}lp{3.0cm}|}

\begin{tabular}{lp{5.3cm}lp{3.0cm}|}
A = input() & Aguarda a entrada de caracteres armazenados em A \\
\end{tabular}
\begin{tabular}{lp{4.7cm}lp{3.0cm}|}
B = int(input()) & Aguarda a entrada de inteiros armazenados em B \\
\end{tabular}
\begin{tabular}{lp{2.6cm}lp{3.0cm}|}
A,B = map(int,input().split()) & Aguarda a entrada de inteiros  separados por espaço, armazenados em A e B respectivamente \\
\end{tabular}
\begin{tabular}{lp{3.0cm}lp{3.0cm}|}
input("Pressione ENTER") & Aguarda pressionar ENTER para prosseguir - como não declarou nenhuma variável, não irá gravar nada. \\
\end{tabular}

\vspace{0.0em} % When there are two boxes, some whitespace may need to be added if the one on the right has more content
}

%------------------------------------------------
% Lógica Básica do Python
%------------------------------------------------

\headerbox{Lógica Básica do Python}{name=introduction,column=1,row=0,bottomaligned=objectives}{

%------IF--------
\colorbox[HTML]{CCFFFF}{\makebox[\textwidth-2\fboxsep][l]{\bf - if}}
\begin{itemize}\compresslist
\item if teste:\\
........\# faça algo se teste der verdadeiro\\
elif teste2\\
........\# faça algo se teste2 der verdadeiro\\
else:\newline
........\# faça algo se ambos derem falso
\end{itemize}


%------WHILE--------
\colorbox[HTML]{CCFFFF}{\makebox[\textwidth-2\fboxsep][l]{\bf - while:}}
\begin{itemize}\compresslist
\item while teste:\\
........\# enquanto verdadeiro continue fazendo algo
\end{itemize}


%------FOR--------
\colorbox[HTML]{CCFFFF}{\makebox[\textwidth-2\fboxsep][l]{\bf - for:}}
\begin{itemize}\compresslist
\item for x in sequência\\
........\# enquanto o x estiver na sequência informada\\
........\# faça algo para cada item na sequência\\
........\# a sequência pode ser uma lista,\\
........\# elementos de uma string, etc.

\item for x in range(10)\\
........\# repita algo 10 vezes (de 0 a 9)

\item for x in range(5,10)\\
........\# repita algo 5 vezes (de 5 a 9)
\end{itemize}

\colorbox[HTML]{CCFFFF}{\makebox[\textwidth-2\fboxsep][l]{\bf - Testes Lógicos}}
\linebreak \\
\begin{tabular}{l l}
10 == 10 & retorna: True \\
10 == 11 & retorna: False \\
10!= 11 & retorna: True \\
"jack" == "jack" & retorna: True \\
"jack" == "jake" & retorna: False \\
10 > 10 & retorna: False \\
10 >= 10 & retorna: True \\
"abc" >= "abc" & retorna: True \\
"abc" < "abc" & retorna: False \\
\end{tabular}
}


%------------------------------------------------
% Listas no Python
%------------------------------------------------

\headerbox{Listas no Python}{name=results,column=2,span=2,row=0}{

\colorbox[HTML]{CCFFFF}{\makebox[\textwidth-2\fboxsep][l]{\bf - Listas no Python}}
\linebreak \\
Listas são compostas por elementos de qualquer tipo (podem ser alteradas) \linebreak \\
\begin{tabular}{@{}ll@{}}
\textbf{Manipulação de Listas no Python}\\
\multicolumn{2}{l}{\cellcolor[HTML]{DDFFFF}Criação} \\
uma\_lista = [5,3,'p',9,'e'] & cria: [5,3,'p',9,'e'] \\
\multicolumn{2}{l}{\cellcolor[HTML]{DDFFFF}Acessando} \\
uma\_lista[0] & retorna: 5 \\
\multicolumn{2}{l}{\cellcolor[HTML]{DDFFFF}Fatiando} \\
uma\_lista[1:3] & retorna: [3,'p'] \\
\multicolumn{2}{l}{\cellcolor[HTML]{DDFFFF}Comprimento} \\
len(uma\_lista) & retorna: 5 \\
\multicolumn{2}{l}{\cellcolor[HTML]{DDFFFF}count( item)} \\
\multicolumn{2}{l}{Retorna quantas vezes o item foi encontrado na lista.} \\
cont(uma\_lista('p') & retorna: 1 \\
\multicolumn{2}{l}{Pode ser usado juntamente com a função while para 'andar' pelo comprimento da lista:} \\
while x < len(uma\_lista): & retorna: [3,'p']\\
\multicolumn{2}{l}{\cellcolor[HTML]{DDFFFF}Ordenar - sort()} \\
uma\_lista.sort() & retorna: [3,5,9,'e','p'] \\
\multicolumn{2}{l}{Ordenar sem alterar a lista} \\
print(sorted(uma\_lista)) & retorna: [3,5,9,'e','p'] \\
\multicolumn{2}{l}{\cellcolor[HTML]{DDFFFF}Adicionar - append(item)} \\
uma\_lista.append(37) & retorna: [5,3,'p',9,'e',37] \\
\multicolumn{2}{l}{\cellcolor[HTML]{DDFFFF}Inserir - insert(position,item)} \\
insert(uma\_lista.append(3),200) & retorna: [5,3,200,'p',9,'e'] \\
\multicolumn{2}{l}{\cellcolor[HTML]{DDFFFF}Retornar e remover - pop(position)} \\
uma\_lista.pop() & retorna: 'e' e a lista fica [5,3,'p',9] - remove o último elemento \\
uma\_lista.pop(1) & retorna: 3 e a lista fica [5,'p',9,'e'] - remove o elemento 1 \\
\multicolumn{2}{l}{\cellcolor[HTML]{DDFFFF}Remover - remove(item)} \\
uma\_lista.remove('p') & retorna: [5,3,9,'e'] \\
\multicolumn{2}{l}{\cellcolor[HTML]{DDFFFF}Inserir} \\
uma\_lista.insert(2,'z') & retorna: [5,'z',3,'p',9,'e'] - insere na posição numerada \\
\multicolumn{2}{l}{\cellcolor[HTML]{DDFFFF}Inverter - reverse()} \\
reverse(uma\_lista) & retorna: ['e',9,'p',3,5] \\
\multicolumn{2}{l}{\cellcolor[HTML]{DDFFFF}Concatenar} \\
uma\_lista+[0] & retorna: [5,3,'p',9,'e',0] \\
uma\_lista+uma\_lista & retorna: [5,3,'p',9,'e',5,3,'p',9,'e'] \\
\multicolumn{2}{l}{\cellcolor[HTML]{DDFFFF}Encontrar} \\
9 in uma\_lista & retorna: True \\
for x in uma\_lista & retorna toda a lista, um elemento por linha \\
......print(x) &  
\end{tabular}
%------------------------------------------------
}
\end{poster}
\newpage

%%%%%%%%%%%%%%%%%%%%%%%%%%%%%%%%%%%%%%%%%%%%%%%%%%%%%%%%%%
%%%%%%%%%%%%%%%%%%    SEGUNDA PÁGINA    %%%%%%%%%%%%%%%%%%
%%%%%%%%%%%%%%%%%%%%%%%%%%%%%%%%%%%%%%%%%%%%%%%%%%%%%%%%%%

\begin{poster}
{
headerborder=closed, colspacing=0.8em, bgColorOne=white, bgColorTwo=white, borderColor=lightblue, headerColorOne=black, headerColorTwo=lightblue, 
headerFontColor=white, boxColorOne=white, textborder=roundedleft, eyecatcher=true, headerheight=0.1\textheight, headershape=roundedright, headerfont=\Large\bf\textsc, linewidth=2pt 
}
%----------------------------------------------------------------
%	TITLE SECTION 
%----------------------------------------------------------------
{\bf\textsc{Python Cheat Sheet}\vspace{0.5em}} % Poster title
{\textsc{\{ P y t h o n \ \ \ \ \ C h e a t \ \ \ \ \ S h e e t\} \hspace{12pt}}}
{\textsc{Michelle Cristina de Sousa Baltazar \\ (Universidade Federal do Triângulo Mineiro) \hspace{12pt}}} 

%----------------------------------------------------------------
%	Outros Elementos
%----------------------------------------------------------------
\headerbox{Outros Elementos}{name=method,column=0}{

\colorbox[HTML]{CCFFFF}{\makebox[\textwidth-2\fboxsep][l]{\bf - Palavras-Chave}}
\begin{tabular}{lp{5.8cm}lp{1.0cm}}
{\bf Oper.} & {\bf Descrição}\\
print & Imprime para a tela \\
while & "Enquanto" - laço para repetição de alguma condição \\
for & "Para" - loop para repetição de alguma condição \\
break & Interrompe o loop caso necessário \\
continue & Interrompe o loop atual sem sair do loop, reiniciando \\
if & "Se" - usado para testar alguma condição \\
elif & É uma variante para o "senão" - se a primeira condição falha, testa a próxima \\
else & "Senão" - é opicional e será executado quando a primeira condição falhar \\
is & Testa a identidade do objeto \\
import & Importa outros módulos para dentro de um script \\
as & Usado para dar um apelido (alias) para um módulo \\
from & Para importar uma variável especifica, classe ou função de um módulo \\
def & Usado para criar uma função nova definida pelo usuário \\
return & Sai da função e retorna um valor \\
lambda & Cria uma função nova anônima \\
global & Acessa variáveis definida globalmente (fora de uma função) \\
try & Especifica manipuladores de exceções  \\
except & Captura a exceção e executa códigos \\
finally & É sempre executado no final, utilizado para limpar os recursos \\
raise & Cria uma exceção definida pelo usuário \\
del & Deleta objetos \\
pass & Não faz nada \\
assert & Usado para fins de depuração \\
class & Usado para criar objetos definidos pelo usuário \\
exec & Executa dinamicamente um código Python \\
yield & É usado com geradores
\end{tabular}


}

%----------------------------------------------------------------
%	Operadores
%----------------------------------------------------------------
\headerbox{Operadores Python}{name=results2,column=1}{
Tomemos como exemplo a=10 e b=20:\\
\colorbox[HTML]{CCFFFF}{\makebox[\textwidth-2\fboxsep][l]{\bf - Operadores Aritméticos}}
\begin{tabular}{lll}
{\bf Op.} & {\bf Descrição} & {\bf Exemplo} \\
+ & Adição & a + b retorna: 30 \\
- & Subtração & a - b retorna: -10 \\
* & Multiplicação & a * b retorna: 200 \\
/ & Divisão & b / a retorna: 2 \\
\% & Módulo & a \% b retorna: 0 \\
** & Exponencial & a**b retorna: $10^{20}$ \\
// & Divisão Piso & 9 // 2 retorna: 4
\end{tabular}

%----OPERADORES DE COMPARAÇÃO-----------
\colorbox[HTML]{CCFFFF}{\makebox[\textwidth-2\fboxsep][l]{\bf - Operadores de Comparação}}
As operações básicas de comparação podem ser usadas de diversas maneiras para todos os tipos de valores - números, strings, sequencias, listas, etc. O retorno será sempre True ou False.\\
\begin{tabular}{lll}
{\bf Op.} & {\bf Descrição} & {\bf Exemplo} \\
< & Menor que & a < b retorna: True \\
<= & Menor ou igual & a <= b retorna: True \\
== & Igual & a == b retorna: False \\
> & Maior que & a > b retorna: False \\
>= & Maior ou igual & a >= b retorna: False \\
!= & Diferente & a != b retorna: True \\
<> & Diferente & a <> b retorna: True
\end{tabular}

%------OPERADORES LÓGICOS-----------
\colorbox[HTML]{CCFFFF}{\makebox[\textwidth-2\fboxsep][l]{\bf - Operadores Lógicos}}
Os operadores lógicos {\bf and} e {\bf or} Também retornam um valor booleano quando usado em uma estrutura de decisão.\\
\begin{tabular}{lp{6.5cm}lp{1.0cm}|}
{\bf Op.} & {\bf Descrição}\\
and & Se o resultado de ambos operadores é verdadeiro, retorna: True \\
or & Se um dos resultados retorna verdadeiro, retorna: True \\
not & É utilizado para reverter o estado lógico de qualquer operação booleana.
\end{tabular}

%-------TUPLAS---------------------
\colorbox[HTML]{CCFFFF}{\makebox[\textwidth-2\fboxsep][l]{\bf - Tuplas no Python}}
Tupla é uma lista de valores separados por vírgulas - é similar à uma lista porém é imutável: \\ 
uma\_tupla = 'a','b','c','d','e' \\
outra\_tupla = ('a','b','c','d','e') \\

%------NÚMEROS ALEATÓRIOS----------
\colorbox[HTML]{CCFFFF}{\makebox[\textwidth-2\fboxsep][l]{\bf - Números Aleatórios}}
Strings são compostos de caracteres: \\ 
uma\_string = "Hello World!" \\
outra\_string = 'Ola Mundo!"

}

%----------------------------------------------------------------
%	Strings no Python
%----------------------------------------------------------------
\headerbox{Strings no Python}{name=conclusion,column=2,span=2,row=0}{
string é uma sequencia de caracteres geralmente usada para armazenar texto. \\
Strings são compostos de caracteres (não podem ser alterados - são imutáveis) \\ 
\linebreak
%% {lp{5.8cm}lp{1.0cm}|}
\begin{tabular}{@{}lp{10.2cm}l@{}}
\multicolumn{2}{l}{\cellcolor[HTML]{DDFFFF}Criação} \\
uma\_string = "Hello World!" & outra\_string = 'Ola Mundo!" \\
\multicolumn{2}{l}{\cellcolor[HTML]{DDFFFF}Acessando} \\
uma\_string[4] & retorna: 'o' \\
\multicolumn{2}{l}{(este caso retorna a 4ª posição do texto - começando a contar a partir do zero)} \\
\multicolumn{2}{l}{\cellcolor[HTML]{DDFFFF}Dividindo} \\
uma\_string.split('') & retorna ['Hello','World'] \\
\multicolumn{2}{l}{(este caso divide o texto no espaço em branco em uma lista de duas strings)} \\
uma\_string.split('r') & retorna ['Hello Wo','ld'] \\
\multicolumn{2}{l}{(este caso divide o texto na letra 'r' em uma lista de duas strings)} \\
\multicolumn{2}{l}{\cellcolor[HTML]{DDFFFF}Unindo} \\
\multicolumn{2}{l}{Para unir uma lista de strings usaremos a função join() } \\
\multicolumn{2}{l}{uma\_lista = ["isto","eh","uma","lista","de","strings"]} \\
' '.join(uma\_lista) & retorna: "isto eh uma lista de strings" \\
' 'TESTE'.join(uma\_lista) & Retorna: \\
''.join(uma\_lista) & retorna: "istoehumalistadestrings" \\
\multicolumn{2}{l}{\cellcolor[HTML]{DDFFFF}Formatando Strings} \\
\multicolumn{2}{l}{Podemos usar o operador \% para adicionar elementos em uma string:} \\
\multicolumn{2}{l}{esta\_string = "todos"} \\
print("Olá para \%s!" \%esta\_string) & retorna: "Olá para todos!" \\
\end{tabular}
\linebreak

%-----OPERAÇÕES COM STRINGS----------
\colorbox[HTML]{CCFFFF}{\makebox[\textwidth-2\fboxsep][l]{\bf - Operações com Strings}}
Definindo as variaveis de string para exemplo da seguinte forma: a = ['Hello'] e b = ['Python'] \\
\begin{tabular}{lp{9.0cm}lp{1.0cm}lp{1.0cm}} %\begin{tabular}{lll}
{\bf Oper.} & {\bf Descrição} & {\bf Exemplo} \\
+ & Concatenation - soma o conteúdo das duas strings & a + b retorna: HelloPython \\
* & Repetition - repete o conteúdo da string N vezes & a*2 retorna: HelloHello \\
.[ ] & Slice - fatia retornando o caractere no respectivo indice & a[1] retorna: "e" \\
.[ : ] & Range Slice - retorna os caracteres do intervalo indicado & a[1:4] retorna: "ell" \\
in & Membership - se o caractere existe na string, retorna true & H in a will give 1 \\
not in & Membership - se o caractere não existe na string, retorna true & M not in a retorna: 1 \\
\% & Format - formata uma string & exemplos na tabela seguinte \\
\end{tabular}

%------FORMATAÇÃO DE STRINGS---------
\colorbox[HTML]{CCFFFF}{\makebox[\textwidth-2\fboxsep][l]{\bf - Formatação de Strings}}
\begin{tabular}{ll|ll} %\begin{tabular}{lp{3.0cm}lp{4.0cm}lp{2.0cm}lp{2.0cm}|} 
{\bf Símbolo} & {\bf Conversão} & {\bf Símbolo} & {\bf Conversão} \\
\%c & caractere					& \%i & decimal inteiro com sinal \\
\%d & decimal inteiro com sinal	& \%u & decimal inteiro sem sinal \\
\%o & octal inteiro				& \%x & hexadecimal inteiro (letras minúsculas) \\
\%f & numero real ponto flutuante & \%X & hexadecimal inteiro (letras maiúsculas) \\
\%g & o menor entre \%f e \%e & \%e & notação exponencial (com 'e' minúsculo) \\
\%G & o menor entre \%f e \%E & \%E & notação exponencial (com 'E' maiúsculo) \\
. & . & \%s & converção de string via str() antes de formatar \\
\end{tabular}

%-----------------------------------
}
%----------------------------------------------------------------
%	REFERENCES  {name=objectives,column=0,row=0}
%----------------------------------------------------------------
%\headerbox{bb}{name=references,column=1,row=0}{}
%----------------------------------------------------------------
%	FUTURE RESEARCH
%----------------------------------------------------------------
%\headerbox{aa}{name=futureresearch,column=1,row=0}{}
%----------------------------------------------------------------
%	CONTACT INFORMATION
%----------------------------------------------------------------
%\headerbox{Contact Information}{name=contact,column=2,span=2,row=0}{}
%----------------------------------------------------------------
\end{poster}
\end{document}